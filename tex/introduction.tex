\section{Introduction}

\subsection{Background}
As part of this module, we were entered into an International IT Challenge run by \emph{Atos}.  Atos publicise this competition to many universities where students will attempt to create a solution to the requirements set by Atos.  This year the challenge revolved around \emph{Connected Living}.  There are many definitions for connected living.  Some definitions define it as something constrained to the home.  Others describe it as a world where customers use different devices to experience connection anytime, anywhere.  This definition covers connected homes, connected work and connected city.  Atos believe that connected living involves bringing the home, workspace and city seamlessly together through smart devices providing connectivity anytime, anywhere.  Consumers are always wanting to feel more connected to their workspace, their homes and their cities.  \\
Originally, there were a few objectives set by Atos that they wished to be fulfilled:
\begin{itemize}
\item Easy Connection of Products
\item Big Data Analytics
\item Create an Audience
\item Client Facing App
\end{itemize}
We tried to think outside the box for this project and decided not to create an app that would deal with the ``connected home'' but something that has never been created before and something we feel still meets the criteria laid out by Atos.  As a team, we wanted to create something that would enhance peoples social lives and provide a service both to the customers and the businesses involved.  

\subsection{The Customer}
From \textbf{Atos}, we are assigned a mentor who acts as a customer.  We have been allocated \textbf{Mike Smith}; a Chief Technology Officer within Atos.  Mike has extensive knowledge of all things technical and has recently written a paper about \emph{Connected Train}.  This makes him an ideal customer as he has a good knowledge of the \emph{Connected World}.  We aim to tap into this knowledge and gain advice and feedback in order to improve and enhance our idea from the beginning until the end of this module.  \\
On another level, when we speak of customers thorough this project documentation, we will not only be referring to Mike, but also the thousands/millions of potential users for this system if and when it is launched.  

\subsection{The Team}
We have decided to call our team \textbf{Hydra}.  As all three of us our MCU (Marvel Cinematic Universe fans), we looked there for inspiration.  At the heart of Marvel is an organisation striving for world domination.  We aim to do the same with our idea and therefore we have decided to call our team after the name of this organisation; Team \textbf{Hydra}.  The team includes three Computer Science Masters students; Jay Vagharia, Oliver Woodings and Simon Kerr.  Being computer scientists, we have a high interest in technology (with a small geek streak) and a passion for music.  To combine both together would allow us to be involved in a project that we are all passionate about and will ensure we are completely motivated to succeed.  We have all successfully completed industrial placements, with one team member already working for another company.  The experience we have between the three of us should provide us with the necessary skills to tackle and complete this project. 

\subsection{The Idea}
In order to create something new, we steered away from the connected home.  We wanted something different, something unique and that took us towards music.  So we started off with trying to connect people with music.  We want to provide the ability to connect to and control the music in your surroundings, anywhere.  We want users to walk in to a coffee shop, library, bar, restaurant etc. and have the ability to suggest their favourite songs that are then played through the public audio media.  \textbf{Choona} is an app focused around the `Connected Music'.  Our research has shown there is no music sharing concept that allows people to collaboratively listen, interact and suggest music to people in your surrounding area, home or business through an intelligent, cloud-driven playlist system.  \\
Choona is a public music player that lets you have a say in what you listen to in public, allowing you to suggest songs that you want to be played at your location.  It provides you with the option to like/dislike songs suggested by others, where increased likes on a song will push that song further up the order allowing it to be played sooner.  Furthermore, you can connect through your mobile device allowing you to listen to the music privately (via your headphones).  \\
Choona provides many different features for the businesses involved.  Advertisements can be added in two forms; visual or audio thus improving product/promotional awareness.  Furthermore, it gives the  customer the opportunity of selecting what they want to hear thus keeping them happy.  