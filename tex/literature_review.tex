\section{Literature Review}


\subsection{\textbf{Competitor}}
    We have taken a look at other apps and services on the market and feel that nothing matches our idea. The closest service available is from \textbf{Sonos}. Sonos is a smart system of speakers and audio components that unite your digital music collection in one app and can then be controlled from any device. The rise of digital music has allowed for us to bring our music wherever we go, through media such as iPods, MP3 players etc. However, there hasn't been the same advancements in terms of systems that don't move around. `Wireless' is the word that comes to mind when thinking of a solution. It is now possible to stream audio to a wireless device (speaker) and without compromising on the sound quality. Sonos provides the user with the ability to play music from a device wirelessly anywhere in the \textbf{home}. However, there are two issues with this; it is only available for the \textbf{home} and the consumer needs to purchase expensive \textbf{hardware}. The cheapest speaker available for purchase is \pounds169. If you are a music-orientated person, you may wish to purchase their high-end hardware which can cost up to \pounds1200. Unfortunately, Sonos does not support other wireless speakers. An adaptor can be purchased to allow these to be connected to their system, the \emph{Connect} device, but this costs \pounds279.  

    \textbf{Pure} have also moved into this market where they provide wireless speakers and hardware for \emph{wireless music} in the \textbf{home}. They also allow you to purchase hardware to link your current speakers with their system at \pounds69.  This is much cheaper than Sonos but is still quite expensive. It allows the consumer to wirelessly play their music from any music app or streaming service they want. \textbf{Bose} also provide a very similar service to Pure, but the hardware costs are more expensive.

    From this, we can see that there are no services available for the wireless sharing of music outside the home. Our app would unite music into anybodys daily routine, whether this is at the office, coffee shop, restaurant as well as the home. We also want to make sure that no expensive costs are applied.
    Competitors can appear at anytime during the development of a project, so it is important that we keep looking for emerging competitors and that we can identfy how our product is unique to theirs. 

\subsection{\textbf{Music Sources}}
    In today's maket, there are many music sources available to an individual. We have virtual music from services such as `Spotify', `Google Music' and `iTunes' as well as physical music on `iPods', `MP3 players' and other hardware devices. 

    \textbf{Spotify} is a music streaming service that offers access to a library of over 20 million music tracks with over 40 million active users. It is available across 58 markets including the UK, USA, France, Germany, Hong Kong and Argentina. It is available on iOS, Android, Windows phone as well as PC and Mac. One chain that is affiliated with Spotify is Costa. Costa have their own playlist that people can access from their device. 
    \textbf{Google Music} is another streaming service and offers the same service as Spotify. Again, they have a large library of songs (around 18 million) and the service is available in over 57 countries on all Android devices as well as web browsers. 
    A year ago, Apple's \textbf{iTunes} accounted for 75\% of the digital music market and with a huge 575 million active users. Although this may have decreased slightly in the last year, that is still a large user base. As well as general users, Starbucks is affiliated with iTunes and use this service to hand-pick and play music throughout their stores. The idea of allowing customers to put forward their music preference may be of interest to a chain like Starbucks amongst others. 

    The above figures suggest that the music streaming industry is vast and that music is a part of many people's lives. Coffee shops have integrated these sources and music into their environment, but without the customer interaction. Choona would provide this interaction. Over the course of this project, we shall identify more sources because having more sources creates a larger user base as well as a better music library. We shall look at how the different sources work to try and make sure we have adaptors in place that can cover the wide variety of sources available. 

\subsection{\textbf{Legal Issues}}
    Music playing for customers or staff through media such as radio, MP3, TV etc. is considered a \emph{public performance}. The \emph{Copyright, Designs and Patents Act 1988} means that an agreement is needed from the copyright owner before the material can be played in public. A music license (PPL) will grant this agreement. In most cases, a license is required but there are a few instances when one is not required. One example of this is where PRS  artists have waived their rights. PRS for Music represents the rights of over 100,000 artists in the UK.  It provides licensing to organisations to allow the playing, performing and availability of copyright music on behalf of the artists and overseas societies.  The royalties are distributed fairly and efficiently.  Another example is a hotel, guest house or B\&B that has fewer than 25 rooms with no areas open to non-residents. 
    Any business such as a coffee shop, bar or gym that plays recorded music in public will legally require a PPL. The likelihood of our service being used in places that don't have a PPL and require one is small. Most coffee shops, restaurants, gyms etc. will already have the license in place. It will be work places deciding to implement our service that will have to go about retrieving a PPL. As part of this section, we shall look at the process of obtaining a PPL, the costs involved and potential constraints.