\section{Conclusion}

We believe the success of a project depends on the preparation of the individuals involved, including but not limited to:
\begin{itemize}
  	\item All members of the team playing to their strengths and identifying weaknesses
	\item Good risk management and active contingency plans
	\item Clear project requirements
	\item Always planning ahead in terms of resources, time and capacity
	\item Effective communication between the team
	\item Active communication with the client
\end{itemize}

From the very beginning of the project we all identified our particular strengths and weaknesses. Any weaknesses were covered by other members of the team in order to prevent an individual's weakness affecting the progress of the project as a whole. Using this information, the roles were assigned to everyone, ensuring all members knew what they were doing in the team. By the end of the project we were performing very efficiently since everyone was working on tasks assigned to their strengths. Furthermore, we had weekly meetings to ensure everyone was on track for the current milestone and to offer help to anyone that was struggling with their tasks. This did not mean that communication was restricted to the weekly meetings; if any suggestions or concerns needed to be raised we communicated through our WhatsApp group (messaging application) as we saw fit. We also had a single point of contact in the team with the client in order to prevent confusion. Any information received from the client was then distributed to the rest of the team accordingly. 

We found using Rapid Application Development throughout the project was very effective. With the frequent contact by the client, we believe we truly benefited from the short continual iterations that the development model allowed. These short iterations in development helped us to be more agile as we could quickly take the feedback from the client and adapt the solution as necessary without having to go through the entire process all over again. A good example of this workflow in practice was when we had a meeting with the client during the very early stages of the project; at that point we had a very early UI prototype to get some feedback while the backend was being developed. This meeting was very influential as he shared particular additions to the UI such as the advertisement banner on the playlist page which added another revenue channel to the application. We were able to prototype this suggestion very quickly in our next development cycle; if we were using a different less flexible development model we would not have been able to act so quickly. RAD also played well with our decision to structure the Choona ecosystem using microservices. As soon as an update was ready for a service it could be rapidly integrated, tested and deployed without having dependencies on other parts of the system.

Looking back to our resource planning, all the resources required for this project were easily accessible. The project overall was not very dependent on hardware other than the Raspberry Pi which was provided by the department for prototyping purposes already. Due to the predominantly software-based nature of the project, version management was used to allow us to roll back changes very quickly if a RAD iteration introduced bugs. It also helped the team track different functionality additions as version increments. For this, the team used the git-based SaaS platform GitHub. The repository did not just help us have a constant backup of our work; GitHub pushed the team to work in a more modular fashion. Features were implemented by individuals on separate branches, which were then peer reviewed by the rest of the team before being merged back into the master branch of the project. This helped keep our RAD cycles fast and bug-free and would not have been possible without the tools available in GitHub.

Reflecting more on the prototype, the chosen application frameworks used would not be appropriate if Choona was to be taken to an enterprise level. Cordova and Phonegap in general are good especially if quick prototyping is needed, however looking towards the future a more native approach might be required for better performance. This would be introduced in phases dependent upon the resources available (different developers might be needed for different mobile platforms) and the market need from the stakeholders (a need to scale the application enough where the application features demand native performance). A problem in particular was with the Ionic framework. Ionic is very new and currently had just gotten out of beta. This raised two problems in particular; the existence of a number of bugs in the framework which are still not resolved and the lack of support for the framework as it's not been fully released yet. In addition none of the team members had experience with AngularJS which resulted in an initial time-consuming learning curve. If we do the project again we would not choose AngularJS as a framework and instead would most likely use ReactJS; this has a steeper learning curve but is a better solution for getting the app to an enterprise level in terms of scalability. The lead programmer in the team already has experience with ReactJS meaning we would have had a running start in terms of developing the prototype. This would mean the rest of the team would have to learn ReactJS which would require an initial time investment, however we believe the benefits would soon make it worthwhile.

In relation to the risk assessment carried out, there were not any triggers that were active over the project duration. There were times over the development phase where the app would sometimes lose functionality because of a conflicting change, however because we used GitHub to manage our code this was not a problem and it was easily fixed by reverting the app back to its last working state. Secondly there were times where the app prototype would crash and lose its playlist data (this occurred in the demo) however because of the microservices architecture, the app seamlessly restarted the appropriate services without any action being required. There was a loss of data here (the songs that the user suggested) but if we had more development time then songs would have been stored in persistent storage instead of in memory, meaning the app would not only seamlessly recover itself but would also no longer suffer from data loss.

There are still several hurdles to overcome in order to get Choona to market, however overall the project prototype was a success, proving that our idea is unique and has significant value.