\section{Idea Analysis}
In November, we put forward the idea of Choona in the PID.  Over time, details have changed and in the section we want to analyse the original idea, the components that make it up and discuss whether or not they remain the same and if not, why have they changed.  

\subsection{The Idea}
We briefly mentioned what Choona was in the introduction.  Its important that we provide a more detailed explanation of the idea and the system behind it.  In order to do this, we will break each section of the project down and briefly explain it.  \\
\textbf{Choona} is the name of our crowd controlled juke box.  It allows users with a mobile device to connect to a Choona location and decide what music they want to listen to.  
\textbf{Login} - the user will be asked to sign into Choona.  There will different options available to the user; they can log in via a username and password or they can login via a social media account (Facebook, Google+, Twitter etc.).  In essence, this opens up the option for more users to connect quickly and easily with Choona and the opportunity to post on their social media account that they are using this app.  \\

The main reason behind this app is to play the music you want.  Therefore in order to do this, we need a playlist.   This playlist will have several pieces of functionality.  First of all, this will contain the list of songs that have been currently added to the playlist.  The user will be able to scroll this page and look at the different songs.  They can then \textbf{up-vote/down-vote them}.  The idea of the up-vote/down-vote is to push songs further up the playlist so they are played quicker.  Lets consider the example; there is a song that I really like and many other users within the facility also like that song.  If we all decide to up-vote that song, then it will move further up the list because it will have received more votes than other songs on the playlist.  The same is true for the other way; if we down-vote the song, it will move further down the playlist.  This page will allow the user to add songs to the playlist.  There will be \textbf{search} functionality; this explores any connected music source(s) and looks for the song name, artist or album depending on the users input.  Any results will be displayed and the user can add the song they want.  As this is a public system, there will need to be considerations taken on the available music.  There may be occasions when we do not want explicit versions of songs being played (especially when there is a risk of children being present - nightclubs may wish to allow for explicit versions as their customers will be over the age of 18).  Depending on the business and their choice; this will be reflected in the search results and the songs played.\\
In order to stop abuse of the system, there will be a limit on the number of songs a user can add to a playlist at any one time.  There are different options available for this at but we are leaning towards a time-limited based approach.  This would mean that a user can only add e.g. one song every fifteen minutes.  It has been suggested that we add a points system to this facility that allows you to gain points through the number of up-votes any of your previously added songs receive.  Once you get over a certain amount of points, the number of songs you can listen to is bumped up.  
There may be instances when no songs have been added to the playlist by users.  If this is the case, there will be a \textbf{default playlist} available that kicks in when the user playlist becomes empty.  This means that there won't be any stage when there is no music being played.  This default playlist is created and maintained by the business through their Choona account.  Now we have the playlist, how do we connect to it?\\

Different locations will have different playlists.  In order to differentiate between these distinct locations, we will make use of \textbf{Geolocation}.  There are two sides to this; the business will have to have to create a boundary - this would be their shop floor area.  We call this a \emph{Geofence} and it acts like a virtual barrier.  When inside the barrier, the user will have access/connection but when they leave, the user looses their access/connection.  The second part to this is the mobile device.  Using the locations services available on mobile devices, we can then determine whether they are inside the geofence.  This therefore means we can allow customers to connect to that locations playlist where they can then add their own song choices and up-vote/down-vote other song choices on the playlist.  
Using this functionality, we can also make sure the state of the playlist is ideal for the current users.  If a user leaves the geofence; any song(s) they have added to the list can be removed if that song(s) has not received any up-votes from any other users. \\

As we mentioned earlier, Choona allows for social media login but we take this one step further.  We want to use social media as a way of socialising with friend.  We have decided to use a notification system where users can add posts from Choona onto their social media account.  A post will contain location (through geolocation), the song currently being played and a timestamp; something like ``Chris is listening to Happy by Pharrell Williams at Loughborough Students Union - 10 minutes ago''.    In the app itself, any notifications posted by friends that are Choona users will be displayed.  This is a small simple but fun aspect to the app and has been introduced because teenagers and young adults in todays society are heavy social media users with a large social influence.  \\

Within the app, there will be a \textbf{history} page.  Within this history page, there will be a list of different Choona locations that you have connected to.  Behind each of these locations, there will be a list of songs that were played from when you entered that geofence to when you left the geofence.  With this functionality, we are trying to provide the user with the ability to check back and find a song that they really liked but do not know the name of.  It is a regular occurrence they we listen to a song but do not know what it is.  With background music, it is hard to use an app like ``Shazam'' or ``Soundhound'' to identify the song for two reasons; the volume is too low or there is too much noise to identify the song.  Therefore the history feature on Choona can allow the user to find out what that song is.  Further to this, there will be functionality to \textbf{purchase or play} that song.  Choona will look at the different music providers on the mobile device and then offer the option for the user to connect to that provider and play/buy the song e.g. if running Spotify, the app will offer the option to ``Play in Spotify'' thus allowing the user to add that song to their own private music collection.  \\

There may be environments when you want to listen to the music but there is too much background noise.  This can be common for customers in coffee shops who are there to work or for people in an office environment.  Choona will enable the user to listen to the playlist privately; through the use of headphones.  Very simple and could be very effective.  \\

There would be two different types of Choona accounts; standard user and admin.  The customer will sign up for the standard user account, providing them with the functionality described above.  The admin account is the account provided to the business.  This is used for the purposes of identifying the music sources.  The admin account will link up with the different available music sources to that business.  This is linked to the search functionality in the customer app so the search results that appear for the customer relate to the music available from the connected source.  Furthermore, the admin account will deal with adverts.  The Choona app allows for advertising in two forms.  The first within the app; at the top of the playlist, there will be an accordion.  Within this accordion, the admin can either have one single image; depicting the business or they could utilise it for the purposes of adverts.  They can place multiple different images that scroll through with each one depicting whatever they want; a new product or a special offer.  The second form of advertising comes in sound bites.  These sound bites would be positioned between songs (this is handled automatically by the Choona system).  The admin account will just have to add the sound bites they want to placed between the songs.  

\subsection{User Value Proposition}
There are four small sections that curtail a value proposition.  These have been outlined below:
\begin{itemize}
\item \textbf{(1)} Headline - one short sentence about the end-benefit we are offering
\item \textbf{(2)} Small paragraph - explanation of what we offer, for whom and why its useful
\item \textbf{(3)} Key benefits and features of the app
\item \textbf{(4)} One visual - pictures paint a thousand words
\end{itemize}
Based on the criteria above, we have written the follow proposition.\\

\textbf{(1)} Choona is a music-sharing app allowing users to collaboratively listen, interact and suggest music in public areas, businesses or homes through an intelligent, cloud-driven system.  \\

\textbf{(2)} This app is for anyone who wants a say in the music they listen to; users can access music playlists through geolocation and add the songs (from different sources) they want.  Songs move further up the playlist the more ``up-votes'' they receive or further down the queue the more ``down-votes'' they receive.  Choona offers social media interaction and provides a catalog of history to help find the song(s) thats name you can't remember.  \\

\textbf{(3)} There are several key benefits to Choona for the user:
\begin{itemize}
\item Listen to your preferred music - In too many situations, we are forced to listen to music that doesn?t interest us.  Though the majority of times the music is in the background, we somehow know it?s there.  If we enjoy that music or can relate to that music in any way, it has a positive effect on our attitude and what we see.
\item You do not need to sign up; you can log in using a social media account.
\item Adverts can help highlight a new range of products or identify any special offers that you may be interested in.
\item Helps identify songs that you cannot remember the name of or the artist that sings it.
\end{itemize} \ 

\textbf{(4)} This image is a...
